\documentclass{article}

\renewcommand{\labelenumii}{\theenumii}
\renewcommand{\theenumii}{\theenumi.\arabic{enumii}.}

\usepackage{graphicx}
\graphicspath{{Figures/}}

\title{Software Requirements Specification}
\date{24 February 2017}
\author{Team Dodger}

\begin{document}
	\pagenumbering{gobble}
	\maketitle
	\newpage
	\tableofcontents
	\newpage
	\pagenumbering{arabic}
	
	\section{Introduction}
	\subsection{Purpose}
	The software requirements specification should aptly outline the functional requirements of the system to ensure that a third party could develop the functionality to a required degree without further input. Thus, the functional requirements should be precise and extensive to eliminate deviation from the system’s goals.
	
	\subsection{Scope}
	The intended audience of the application includes students of the University of Pretoria, the staff and simple visitors. The NavUP system will help users to navigate campuses by allowing users to choose destinations, locate their current location, set up the appropriate path by taking into account human congestion and visually representing said path for the user to follow. NavUP will also have a notifications system that will tell the user of events he/she might be interested in. An achievements system will also be in place to award users for walking certain distances or visiting certain locations. NavUP can be used by new students who do not know their way around campuses yet, or simply by staff members who want to avoid clustered pathways. It is a tool that will help optimise campus navigation and reduce travel time from one destination to the other. NavUP also hopes to incorporate locations with access for the disabled into its maps for those that are in need of such features. The application must be functional both online and offline. 
	
	\subsection{Definitions, Acronyms and Abbreviations}
	\subsection{References}
	\subsection{Overview}
	The SRS will help give a detailed representation of the functional requirements and how they will work together to make up NavUP. 
	
	\section{Overall Description}
	\subsection{Product Perspective}
	\subsection{Product Function}
	\subsection{User Characteristics}
	\subsection{Constraints}
	\subsection{Assumptions and Dependencies}
	
	\section{Specific Requirements}
	This section expands on the functional requirements of the system. It gives a detailed 	description of the system and all of its use cases.
	
	\subsection{External Interface Requirements}
	
	\subsection{Functional Requirements}
	This section includes all functional requirements in detail. It includes all use case diagrams, Actor-System interaction diagrams as well as a traceability matrix.	
	
	\subsubsection{Use cases}
	\begin{enumerate}
		\item \underline{Navigation Subsystem}
			
		
	\begin{enumerate}
		\item Get current location
		\begin{enumerate}
			\item \textbf{Description:} The NavUP system must be able to determine a user’s location at any point in time while the user is on the Hatfield campus. The location must be determined regardless of whether the user is indoors or outdoors.
			\item \textbf{Precondition:} The user must have an active account and must be within range of WiFi routers.
			\item \textbf{Postcondition:} The user’s location is determined and displayed.\newline
		\end{enumerate}
		
		\item Search location
		\begin{enumerate}
			\item \textbf{Description:} The NavUP system must provide functionality that enables a user to search for any location (lecture hall, day-house, restaurant) on the Hatfield Campus.
			\item \textbf{Precondition:} The user must have an active account
			\item \textbf{Postcondition:} Matching locations are returned to the user. If no buildings match the search criteria, an appropriate error message is displayed.\newline
		\end{enumerate}
		
		
		\item View location details
		\begin{enumerate}
			\item \textbf{Description:} The NavUP system must allow users to view details related to specific locations. This could include restaurant menus, lecture hall timetable schedules as well as images of the buildings.
			\item \textbf{Precondition:} The user must have an active account and a valid location must be selected on the map.
			\item \textbf{Postcondition:} Relevant location details shown to user.\newline
		\end{enumerate}
		
		\item View places of interest
		\begin{enumerate}
			\item \textbf{Description:} The NavUP system must be able to display places of interests to a user based on their current location. This will include places like restaurants and day-houses that must be displayed in a list form. 
			\item \textbf{Precondition:} The user must have an active account and their current location must be known.  
			\item \textbf{Postcondition:} Relevant places of interest are listed and displayed to the user based on their location.\newline
		\end{enumerate}
		
		\item Navigate to location
		\begin{enumerate}
			\item \textbf{Description:} The NavUP system must be able to provide directions and navigate to a location given the user’s current location as well as a desired destination. The system should calculate the most optimal route by looking at the shortest path as well as pedestrian traffic.
			\item \textbf{Precondition:} The user must have an active account. The user’s current location must be known and the must have specified a destination through the search interface.
			\item \textbf{Postcondition:} The user is provided with directions from their current location to their desired destination.\newline
		\end{enumerate}
		
		\item Show pedestrian traffic
		\begin{enumerate}
			\item \textbf{Description:} The NavUP system must be able to display pedestrian traffic on campus in the form of a heatmap. When navigating to a specified location, the system must show traffic on that specific route. A user should also be able to view an overall heatmap of the campus to see traffic.
			\item \textbf{Precondition:} Users must all have the NavUP app installed and must be registered in order for them to show up on the heatmap.
			\item \textbf{Postcondition:} A heatmap of the campus is displayed. 
		\end{enumerate}
	\end{enumerate}
	\begin{figure}[h!]
		\includegraphics[scale=0.5]{Navigation_Subsystem.png}
		\caption{Navigation Subsystem}	
	\end{figure}
	
	
	\item \underline{Location Management Subsystem}
	
	\begin{enumerate}
		\item Save Location
		\begin{enumerate}
			\item \textbf{Description:} The NavUP system must be able to save a location that the user specifies. 
			\item \textbf{Precondition:} User must be logged in if they want to save locations, and be in range of WiFi.
			\item \textbf{Postcondition:} A location will be saved to the users profile.\newline
		\end{enumerate}
		
		\item View Saved Location 
		\begin{enumerate}
			\item \textbf{Description:} The NavUP system must allow users to view locations that the user has saved to their profile. 
			\item \textbf{Precondition:} The user must be logged in and be in range of WiFi.
			\item \textbf{Postcondition:} None.\newline
		\end{enumerate}
		
		\item View History 
		\begin{enumerate}
			\item \textbf{Description:} The NavUP system should be able to allow users to view their history of locations.
			\item \textbf{Precondition:} The user must be logged in and be in range of WiFi.
			\item \textbf{Postcondition:} None.\newline
		\end{enumerate}
		
		\item Share Location  
		\begin{enumerate}
			\item \textbf{Description:} The NavUP system should provide the users with a means to share their current location so as to allow other users to know where they are. 
			\item \textbf{Precondition:} The user must be logged in and be in range of WiFi.
			\item \textbf{Postcondition:} User location is broadcasted for other users to see.\newline
		\end{enumerate}
		
		\item View Most Visited Locations  
		\begin{enumerate}
			\item \textbf{Description:} The NavUP system should have a favourite location section which can then be accessed by the user to view their most visited areas. 
			\item \textbf{Precondition:} The user must be logged in and be in range of WiFi.
			\item \textbf{Postcondition:} None.\newline
		\end{enumerate}
		
	\item \underline{User Account Management Subsystem}
	
	
	\item \underline{Entertainment Subsystem}
			\begin{enumerate}
		\item View events
		\begin{enumerate}
			\item \textbf{Description:} The NavUP system must enable users to view all events that are happening around campus in chronological order. The system should suggest events to a user based on their preferences and most visited locations.
			\item \textbf{Precondition:} The user must have an active account and must be logged in.
			\item \textbf{Postcondition:} Various campus-wide events are returned to the user.\newline
		\end{enumerate}
		
		\item Save event
		\begin{enumerate}
			\item \textbf{Description:} The NavUP system must enable users to save events that they are interested so that they can be viewed later.
			\item \textbf{Precondition:} The user must have an active account, must be logged in and there must be events available to save.
			\item \textbf{Postcondition:} An event is saved.\newline
		\end{enumerate}
		
		\item Delete event
		\begin{enumerate}
			\item \textbf{Description:} The NavUP system must enable a user to delete any saved events
			\item \textbf{Precondition:} The user must have an active account, must be logged in and must have saved events
			\item \textbf{Postcondition:} A saved event is deleted .\newline
		\end{enumerate}
	\end{enumerate}
	\begin{figure}[h!]
		\includegraphics[scale=0.5]{Entertainment_Subsystem.png}
		\caption{Navigation Subsystem}	
	\end{figure}	
	
	\item \underline{Achievements Subsystem}
	\item \underline{Administration Subsystem}
	\end{enumerate}
	
	
	\subsection{Performance Requirements}
	\subsection{Design Constraints}
	\subsection{Software System Attributes}
	\subsection{Other Requirements}

\end{document}
